
\section{RELATED WORK}
In this section, we discuss a number of important studies that are related to Peapods.

\textbf{Attack and Defense on Cryptographic Keys.}
Memory disclosure attacks pose a serious threat to cryptographic keys.Harrison et al. provide ways to keep only one copy of cryptographic keys in allocated memory\cite{Harrison2007Protecting}.Scrash removes sensitive data from crash reports in the case of program failures to avoid sensitive data being leaked to disks\cite{Broadwell2003Scrash}.Shreds\cite{Chen2016Shreds} is proposed to limit the memory that each thread can access.It is a general software key protection implementation. Henson et al. use special hardware iRAM to implement full-disk encryption\cite{Henson2013Beyond}.

M{\"u}ller et al. propose TRESOR\cite{Freiling2011TRESOR} to prevent the cold-boot attacks on full-disk encryption, by storing AES keys in registers only.Using the AES key protected by TRESOR as a key-encryption key, PRIME\cite{Garmany2013PRIME} and RegRSA\cite{Zhao2016RegRSA} implemented the RSA computations in registers,RegRSA uses vector instructions in calculations, it provides much better performance than PRIME.  Copker\cite{Guan2014Copker} and Mimosa\cite{Guan2015Protecting} implemented the RSA computations in caches.Mimosa can defend against both software memory attacks and cold-boot attacks due to the use of Intel TSX\cite{extensions}.Ramcrypt\cite{Drescher2016RamCrypt} is a memory encryption implementation. When the program is running, only the page (4KB) that the program is accessing is decrypted. Mashtizadeh et al. propose CCFI\cite{Mashtizadeh2015CCFI} to CFI based on cryptographic message authentication codes and the implementation stores AES keys in user-mode registers.

Intel Software Guard eXtensions (SGX) provide a hardware-enabled secure container that is isolated from other processes\cite{sgx}. Confidentiality and integrity of the protected process will be maintained even in the presence of privileged malware. SGX shows the same tendency and potential as TSX that secure systems can be built on top of hardware features.

\textbf{Transaction Memory Implementation.}
Intel TSX features support in the fourth generation of Core CPU (Haswell architecture), implementing hardware transaction memory mechanism. In addition to the Intel TSX, there are a variety of hardware implementation transaction memory mechanism design\cite{Hammond2004Transactional}\cite{Ananian2006Unbounded}, also included in SUN SPARC\cite{Dice2009Early}, IBM System z\cite{Jacobi2012Transactional}, IBM Blue Gene / Q\cite{Wang2012Evaluation} and other platforms implementations.The transaction memory mechanism can also be implemented by software\cite{Carlstrom2006The}\cite{Harris2005Composable}\cite{Ni2008Design}\cite{shavit1995software}\cite{Schindewolf2009Towards}\cite{Herlihy2006A}, or a mixed implementation of software and hardware\cite{Moore2006LogTM}\cite{Damron2006Hybrid}\cite{Kumar2006Hybrid}\cite{Rajwar2005Virtualizing}: some functions are implemented by CPU hardware, and some functions are implemented by system software.For distributed computing environments, there is also a corresponding distributed transaction memory mechanism\cite{Bocchino2008Software}\cite{Couceiro2009D2STM}\cite{Kotselidis2008DiSTM}\cite{Romano2010Cloud}\cite{Saad2011Snake} for memory read and write control between multiple computers.


\textbf{Transaction Memory Application.}
The transaction memory mechanism directly relates to the read and write control of the memory data. At the same time, a large amount of system security is related to the access of the memory data\cite{Szekeres2013Eternal}.In 2008, TMI\cite{Birgisson2008Enforcing} completed the implementation of the authorization strategy based on the software transaction memory mechanism.TSX-CFI implements coarse-grained control flow integrity and fine-grained control flow integrity based on RTM and HLE, respectively.TxIntro\cite{Liu2014Concurrent} and Mimosa\cite{Guan2015Protecting}  implement memory data security with memory control of transactional memory mechanisms:TxIntro monitors the abnormal changes in the Read-Set detection data in the transaction memory task, and Mimosa monitors the Write-Set to detect unauthorized read operations in the transaction memory task.HAFT\cite{Kuvaiskii2016HAFT} implemented the instruction-level redundancy Instruction-Level Redundancy (ILR) fault-tolerant system by using the transaction memory mechanism's rollback handling.DrK attack\cite{Jang2016Breaking} exploits the application layer isolation feature of the exception handling of the transaction memory mechanism to attack the Kernel Address Space Layout Randomization (KASLR):even if the user-mode application has an arbitrary number of memory access exceptions, it will not fall into the exception handling of the operating system.Using the above features, T-SGX\cite{Shih2017T} implements a defense mechanism against malicious operating systems and eliminates the malicious operating system's Page-fault side channel attack on SGX Enclave\cite{Xu2015Controlled}\cite{Shinde2015Preventing}.
