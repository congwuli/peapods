

\section{EVALUATION}
We conducted an experiment to test the performance of Peapods.The experimental machine has an Intel Core i7-4770S quad-core processor running a Linux operating system (kernel version 3.13.1).We experimented with the PolarSSL cryptographic algorithm library and used Peapods to protect 2048-bit RSA private key calculations after splitting into 128 transactions and 256 transactions.The comparison objects in the experiment include: (1) the official default configuration of PolarSSL (version 1.3.9), (2) Peapods\_128, the Peapods split into 128 transactions,(3) Peapods\_256, the Peapods split into 256 transactions.
\subsection{Local Performance}
The local throughput for PolarSSL, Peapods\_128, and Peapods\_256 are: 391/s, 341/s, and 352/s (4 threads).Compared to PolarSSL, the performance of Peapods\_128 and Peapods\_256 dropped by 12.8\% and 10.0\%, respectively.The performance overhead introduced by Peapods mainly comes from: (1) waste of CPU clock cycles due to rollback of transactions; (2) encryption and decryption protection of sensitive information and intermediate calculation results; (3) preloading.The reason Peapods\_256 performs slightly better than Peapods\_128 is that the more granular splitting reduces transaction abort due to time-outs and other factors.

We also tested whether a memory-intensive program will have a greater impact on Peapods by running the Geekbench 3 memory stress test while Peapods is performing RSA private key calculations.In the experiment, 4 different memory stress tests will run on all CPU cores, which will cause about 10GB/s of memory data transfer.The maximum memory transfer rate supported by the machine when no user program is running is 13.7GB/s.Peapods\_128 dropped from 341 RSA decryptions per second to 212 RSA decryptions per second, performance decreased by 37.8\%, Peapods\_256 decreased from 352 RSA decryptions per second to 218 RSA decryptions per second, performance decreased by 38.1\%.PolarSSL decreased from 391 RSA decryptions per second to 241 RSA decryptions per second, performance decreased by 38.4\%.

In short, the performance overhead of Peapods is acceptable. At the same time, Peapods perform better than other transaction-based protection due to preloaded optimizations.
\subsection{Impact on Concurrent Processes}
We used Geekbench 3 to test Peapods' impact on other concurrent processes.Figure X shows Geekbench 3 scores in multi-core mode and single-core mode.The baseline is a score in clean environment.

The Geekbench 3 benchmark program performed integer, floating point, and memory bandwidth tests, respectively.PolarSSL, Peapods\_128, and Peapods\_256 experiments have similar scores. Unprotected PolarSSL is slightly better than Peapods.When Geekbench occupies multiple cores, the load change that simultaneously processes RSA private key calculation requests cannot be ignored��the baseline has a clear demarcation from other scores.It is worth noting that XMM registers are mainly used for floating-point calculations, and generally only 1-2 XMM registers are used.When the environment has only a small number of floating point calculations, we will not significantly affect the XMM7 register from the register allocation pool. When there are a large number of floating point calculations in the environment, the overall performance will be slightly reduced.In short, Peapods do not have a significant additional impact on other concurrent processes.
\subsection{��SGX�ıȽ�}

