
\section{BACKGROUND}

\subsection{Memory Disclosure Attacks}
%��Ҫ�����ڴ�й©������������в
There are many security threats when the cryptographic algorithm is running. One of them is sensitive information stored in the memory, such as a key, which is stolen by an adversary.One of the attacks is a memory disclosure attack. Memory disclosure attacks are read-only memory attacks where the adversary can obtain the contents of the memory but cannot tamper with it. Memory disclosure attacks include software attacks and hardware attacks.Software attack means that the attacker does not physically touch the attack target. It uses software vulnerabilities to bypass the system protection mechanism and illegally accesses the data in the memory.Such as OpenSSL heart bleeding attack, using OpenSSL heartbeat vulnerability, so that remote attackers can obtain a server 64KB memory data by constructing a malicious request.Hardware attack means that an attacker can physically contact an attack target and use a physical method to read the memory of the target.Such as cold-boot attack, using the delay disappearance effect of DRAM, directly read the contents of the memory chip.In short, memory disclosure attacks pose a serious threat to plain-text sensitive data in memory.

\textbf{physical memory attack}

Physical memory attack refers to the behavior of the attacker directly reading the memory chip and obtaining the data illegally after the attacker has physical control over the attack target.In this case, any software-based protection will fail because the attacker can directly manipulate the hardware.Physical memory attack mainly include two types: (1)cold boot attack, which uses DRAM's delay-disappearing effect to directly read the contents of the memory chip, (2)DMA attack, which initiate illegal DMA requests through malicious peripherals, and directly read data from the memory chip.

\textbf{software memory attack}

Software memory attack refers to the behavior of the attacker does not physically contact the attack target, and uses software vulnerabilities to bypass the system protection mechanism and illegally access the data in the memory.For example, OpenSSL heart bleeding attack exploits the vulnerability of OpenSSL heartbeat to enable remote attackers to obtain 64KB of memory data from the server once by constructing a malicious request.Software memory attack mainly include four types: (1)attack based on an isolation mechanism vulnerability, which exploits the defect of the virtual memory subsystem of the operating system to illegally access data from the memory chip, (2)attack based on unclear dynamic memory, which originates from the fact that the dynamic memory storing sensitive information is not cleared and is used by other programs, (3)attack based on cryptographic software's own vulnerability, such as OpenSSL heart bleeding attack, (4)attack based on memory data diffusion, which exploits normal operating system functionality.For example, kernel dumps were originally used to save memory images to disk for developers to debug after a software crash, but attackers can also use this feature to deliberately trigger an exception that causes the software to crash, and then read the sensitive data contained in the process from the disk.

%These types of attacks pose a great threat to users' sensitive data.We need a complete solution to defend against these attacks and protect users' data security.
%ö���ڴ�й©��������

\subsection{TSX}
%TSX and RTM
%���������ڴ�
Transactional memory is a software technique that simplifies writing concurrent programs.The main idea is to declare a region of code as a transaction. A transaction executes and atomically commits all the results to memory when the transaction succeeds or aborts and cancels all the results if the transaction fails. Transactional memory mechanism provide the Atomicity, Consistency and Isolation qualities. These transactions can safely execute in parallel, and only serial execution is performed in the case of data conflicts, which happen when several threads access the same memory address at the same time and at least one is write operation.When a data conflict occurs, it will cause a rollback and all the results will be canceled.

%TSX
The recent development is Intel��s TSX and the implementation in Haswell.Haswell is the first x86 processor to feature hardware transactional memory.Programmers only need to specify
critical sections for transactional executions, the processor transparently performs data conflict detection, transaction commit and rollback.TSX tracks the read-set and write-set of a transaction, at a 64B cache line granularity. The read-set and write-set are respectively all the cache lines that the transaction has read from, or written to during execution. A transaction encounters a conflict if a cache line in its read-set is written by another thread, or if a cache line in its write-set is read or written by another thread.

%HLE��RTM
TSX provides two interfaces for programmers to make use of transactional memory. The first mode is Hardware Lock Elision(HLE), provides a pair of compiler hints:xacquire and xrelease to enter or exit the critical section.Once abort happens, the processor will roll back to the original state, and then automatically restarts the execution in a legacy manner. The second mode is Restricted Transactional Memory (RTM), provides three new instructions, XBEGIN, XEND and XABORT, to start, commit, and abort a transactional execution.When using the RTM interface, the programmer needs to specify a fallback handler for the XBEGIN instruction as a parameter and write the fallback handler logic. When a transaction aborts, the program jumps to the specified fallback handler and executes the corresponding program logic.
\subsection{LLVM}

LLVM is an abbreviation of low level virtual machine. It is actually a compiler framework. With the continuous development of this project, llvm has been unable to fully represent this project, but this name has been continued, and it can now be understood as a full-featured compiler. LLVM can be seen as a collection of compiler and toolchain technologies, and they are modular and reusable.

The traditional compiler is divided into front-end, optimizer, back-end three stages, llvm is also divided into three stages, but slightly different in the design.

Because LLVM is a virtual machine, it has its own Intermediate Representation. It needs to compile the LLVM byte code into a platform-specific assembly language before it can be run by the native assembler and linker to generate assembly code, executable shared libraries, etc.

The front end fetches the source code and then turns it into an intermediate byte code representation.This translation simplifies the work of other parts of the compiler so that they do not need to deal with all the complexity of C++ source code.

In LLVM Optimizer, PASS transforms the program between the intermediate representation.In general, the process is used to optimize the code, that is, the process output program is completely functionally identical to its input program, but the performance is improved.However, we can also add code logic by adding an IR PASS to achieve a functional improvement over the input program.LLVM provides a lot of APIs for manipulating intermediate bytecode representations, so we can use these interfaces to generate IR directly in memory and run directly to achieve programmatic changes.The intermediate representation of the source code can be further divided into modules, functions, basic blocks, and instruction four-layer structures.

The back-end part translates the intermediate representation into the assembly language of the target platform and generates the actual running machine code.

%Our work is to modify the intermediate representation of the source code through a Module PASS during the LLVM optimizer stage, adding logic such as generating a master key, identifying sensitive variables, and encrypting and decrypting sensitive data variables (unlike other situations. At this time, the program output by the process has changed in function compared with the program it has input), which enables the program to automatically protect the identified sensitive data variables and prevent the leakage of sensitive data information.
%\section{ THREAT MODEL AND security goals}
